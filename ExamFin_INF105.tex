% Options for packages loaded elsewhere
\PassOptionsToPackage{unicode}{hyperref}
\PassOptionsToPackage{hyphens}{url}
%
\documentclass[
]{article}
\usepackage{amsmath,amssymb}
\usepackage{lmodern}
\usepackage{iftex}
\ifPDFTeX
  \usepackage[T1]{fontenc}
  \usepackage[utf8]{inputenc}
  \usepackage{textcomp} % provide euro and other symbols
\else % if luatex or xetex
  \usepackage{unicode-math}
  \defaultfontfeatures{Scale=MatchLowercase}
  \defaultfontfeatures[\rmfamily]{Ligatures=TeX,Scale=1}
\fi
% Use upquote if available, for straight quotes in verbatim environments
\IfFileExists{upquote.sty}{\usepackage{upquote}}{}
\IfFileExists{microtype.sty}{% use microtype if available
  \usepackage[]{microtype}
  \UseMicrotypeSet[protrusion]{basicmath} % disable protrusion for tt fonts
}{}
\makeatletter
\@ifundefined{KOMAClassName}{% if non-KOMA class
  \IfFileExists{parskip.sty}{%
    \usepackage{parskip}
  }{% else
    \setlength{\parindent}{0pt}
    \setlength{\parskip}{6pt plus 2pt minus 1pt}}
}{% if KOMA class
  \KOMAoptions{parskip=half}}
\makeatother
\usepackage{xcolor}
\usepackage[margin=1in]{geometry}
\usepackage{graphicx}
\makeatletter
\def\maxwidth{\ifdim\Gin@nat@width>\linewidth\linewidth\else\Gin@nat@width\fi}
\def\maxheight{\ifdim\Gin@nat@height>\textheight\textheight\else\Gin@nat@height\fi}
\makeatother
% Scale images if necessary, so that they will not overflow the page
% margins by default, and it is still possible to overwrite the defaults
% using explicit options in \includegraphics[width, height, ...]{}
\setkeys{Gin}{width=\maxwidth,height=\maxheight,keepaspectratio}
% Set default figure placement to htbp
\makeatletter
\def\fps@figure{htbp}
\makeatother
\setlength{\emergencystretch}{3em} % prevent overfull lines
\providecommand{\tightlist}{%
  \setlength{\itemsep}{0pt}\setlength{\parskip}{0pt}}
\setcounter{secnumdepth}{-\maxdimen} % remove section numbering
\ifLuaTeX
  \usepackage{selnolig}  % disable illegal ligatures
\fi
\IfFileExists{bookmark.sty}{\usepackage{bookmark}}{\usepackage{hyperref}}
\IfFileExists{xurl.sty}{\usepackage{xurl}}{} % add URL line breaks if available
\urlstyle{same} % disable monospaced font for URLs
\hypersetup{
  pdftitle={ExamFin\_INF104},
  pdfauthor={Peter,Fabrice,Darwin,Bendy},
  hidelinks,
  pdfcreator={LaTeX via pandoc}}

\title{ExamFin\_INF104}
\author{Peter,Fabrice,Darwin,Bendy}
\date{2023-06-17}

\begin{document}
\maketitle

\hypertarget{introduction}{%
\section{Introduction}\label{introduction}}

\begin{verbatim}
## [1] "Nous avons choisi les variables de taux de change, d'exportations, d'importations et d'IDE (investissements directs étrangers) pour analyser la performance économique d'Haïti sur une période de 33 ans, de 1988 à 2021. Le taux de change est un indicateur important pour mesurer la compétitivité de la monnaie nationale sur les marchés internationaux. Avec les importations, le signe attendu est négatif(-) car toute augmentation du taux de change décourage les importations. Les exportations, elles, impactent le taux de change positivement car toute augmentation de celui-ci(dévaluation de la monnaie locale) engendre une augmentation des exportations. Enfin, les IDE sont un indicateur clé des flux de capitaux étrangers entrants mais sachant qu'ils dépendent de la stabilité politique et des règlementations économiques, le signe attendu du taux de change ne sera pas vraiment clair."
\end{verbatim}

\hypertarget{tableaux-des-variables}{%
\section{Tableaux des variables}\label{tableaux-des-variables}}

\hypertarget{tableau-1}{%
\subsubsection{Tableau 1}\label{tableau-1}}

Pays

Date

Tauxdechange

Exportations

Importations

IDE

Haiti

1988

5.000000

16.302132

848904800

0.38639185

Haiti

1989

5.000000

13.015419

753772600

0.34207478

Haiti

1990

5.000000

11.910184

1068364600

0.25837375

Haiti

1991

6.034167

10.311445

1024557089

-0.05182032

Haiti

1992

9.801667

7.571511

563516992

-0.09746927

Haiti

1993

12.822500

10.845350

711986182

-0.14907504

Haiti

1994

15.040000

6.420727

535668263

-0.12917725

Haiti

1995

15.109733

9.133564

807821955

0.26303505

Haiti

1996

15.701150

11.327631

840148096

0.14101391

Haiti

1997

16.654500

10.452542

882865728

0.11979854

Haiti

1998

16.765667

9.900471

991316376

0.28894367

Haiti

1999

16.937892

12.249112

1233633425

0.72224131

Haiti

2000

21.170667

7.347065

1366788598

0.19446465

Haiti

2001

24.429083

7.031211

1316250001

0.06948874

Haiti

2002

29.250483

6.952309

1250030000

0.09408837

Haiti

2003

42.366758

9.732161

1416987900

0.28590207

Haiti

2004

38.352033

8.588606

1562037812

0.09773131

Haiti

2005

40.448550

8.422388

1852887695

0.36191211

Haiti

2006

40.408517

9.165917

2141589441

2.13617581

Haiti

2007

36.861417

8.182514

2384473294

0.78233595

Haiti

2008

39.107592

8.747568

2853795286

0.28420944

Haiti

2009

41.197608

8.916116

2804202714

0.47831275

Haiti

2010

39.797400

8.570288

4287330050

1.50092985

Haiti

2011

40.522822

10.174764

4195339999

0.91476862

Haiti

2012

41.949723

9.655094

4195363034

1.13794468

Haiti

2013

43.462783

10.514402

4442565271

1.08652151

Haiti

2014

45.215981

10.978847

4753280624

0.65392872

Haiti

2015

50.706427

11.783025

4490959841

0.71245803

Haiti

2016

63.335818

11.534050

4691487577

0.74994493

Haiti

2017

64.769680

11.069652

5215383364

2.49315616

Haiti

2018

68.031754

10.816190

5997556607

0.63810259

Haiti

2019

88.814966

11.714119

5536983124

0.50724209

Haiti

2020

93.509807

7.651007

4318064226

0.17231613

Haiti

2021

89.226637

7.114724

6269037074

0.24492952

\hypertarget{nuage-de-points}{%
\section{Nuage de points}\label{nuage-de-points}}

\hypertarget{graph-a-taux_de_changeexportations}{%
\subsubsection{Graph A :
Taux\_de\_change\textasciitilde Exportations}\label{graph-a-taux_de_changeexportations}}

\begin{verbatim}
## `geom_smooth()` using formula = 'y ~ x'
\end{verbatim}

\includegraphics{ExamFin_INF104_files/figure-latex/unnamed-chunk-3-1.pdf}
\#\#\# Graph B : Taux\_de\_change\textasciitilde Importations

\begin{verbatim}
## `geom_smooth()` using formula = 'y ~ x'
\end{verbatim}

\includegraphics{ExamFin_INF104_files/figure-latex/unnamed-chunk-4-1.pdf}

\hypertarget{graph-c-taux_de_changeide}{%
\subsubsection{Graph C :
Taux\_de\_change\textasciitilde IDE}\label{graph-c-taux_de_changeide}}

\begin{verbatim}
## `geom_smooth()` using formula = 'y ~ x'
\end{verbatim}

\includegraphics{ExamFin_INF104_files/figure-latex/unnamed-chunk-5-1.pdf}

\hypertarget{ruxe9gression}{%
\section{Régression}\label{ruxe9gression}}

\hypertarget{tableau-de-regression}{%
\subsubsection{Tableau de regression}\label{tableau-de-regression}}

\begin{verbatim}
## 
## Call:
## lm(formula = Tauxdechange ~ Exportations + Importations + IDE, 
##     data = tableau_combine)
## 
## Residuals:
##      Min       1Q   Median       3Q      Max 
## -16.1788  -7.3525  -0.4275   3.8454  29.9124 
## 
## Coefficients:
##                Estimate Std. Error t value Pr(>|t|)    
## (Intercept)   2.745e+01  9.687e+00   2.834  0.00815 ** 
## Exportations -2.168e+00  9.420e-01  -2.301  0.02850 *  
## Importations  1.237e-08  1.212e-09  10.208 2.82e-11 ***
## IDE          -3.950e+00  3.828e+00  -1.032  0.31039    
## ---
## Signif. codes:  0 '***' 0.001 '**' 0.01 '*' 0.05 '.' 0.1 ' ' 1
## 
## Residual standard error: 11.03 on 30 degrees of freedom
## Multiple R-squared:  0.8165, Adjusted R-squared:  0.7982 
## F-statistic:  44.5 on 3 and 30 DF,  p-value: 3.662e-11
\end{verbatim}

\hypertarget{nuage-de-points-valeurs-ruxe9siduelles-vs-valleurs-estimuxe9es}{%
\subsubsection{Nuage de points valeurs résiduelles vs valleurs
estimées}\label{nuage-de-points-valeurs-ruxe9siduelles-vs-valleurs-estimuxe9es}}

\includegraphics{ExamFin_INF104_files/figure-latex/unnamed-chunk-7-1.pdf}

\hypertarget{commentaires}{%
\subsubsection{Commentaires}\label{commentaires}}

\begin{verbatim}
## [1] "Idéalement, on s'attendrait à ce que les points soient répartis de manière homogène aléatoire autour de l'axe y = 0, ce qui indiquerait que le modèle linéaire est approprié pour expliquer la variation de la variable dépendante.Cependant, il y a une légère tendance à l'hétéroscédasticité, ce qui pourrait être dû à la présence d'observations atypiques dont il faudrait vérifier la source."
\end{verbatim}

\end{document}
